% Options for packages loaded elsewhere
\PassOptionsToPackage{unicode}{hyperref}
\PassOptionsToPackage{hyphens}{url}
\PassOptionsToPackage{dvipsnames,svgnames*,x11names*}{xcolor}
%
\documentclass[
]{article}
\usepackage{lmodern}
\usepackage{amssymb,amsmath}
\usepackage{ifxetex,ifluatex}
\ifnum 0\ifxetex 1\fi\ifluatex 1\fi=0 % if pdftex
  \usepackage[T1]{fontenc}
  \usepackage[utf8]{inputenc}
  \usepackage{textcomp} % provide euro and other symbols
\else % if luatex or xetex
  \usepackage{unicode-math}
  \defaultfontfeatures{Scale=MatchLowercase}
  \defaultfontfeatures[\rmfamily]{Ligatures=TeX,Scale=1}
  \setmainfont[]{cochineal}
  \setsansfont[]{Fira Sans}
  \setmonofont[]{Fira Code}
\fi
% Use upquote if available, for straight quotes in verbatim environments
\IfFileExists{upquote.sty}{\usepackage{upquote}}{}
\IfFileExists{microtype.sty}{% use microtype if available
  \usepackage[]{microtype}
  \UseMicrotypeSet[protrusion]{basicmath} % disable protrusion for tt fonts
}{}
\makeatletter
\@ifundefined{KOMAClassName}{% if non-KOMA class
  \IfFileExists{parskip.sty}{%
    \usepackage{parskip}
  }{% else
    \setlength{\parindent}{0pt}
    \setlength{\parskip}{6pt plus 2pt minus 1pt}}
}{% if KOMA class
  \KOMAoptions{parskip=half}}
\makeatother
\usepackage{xcolor}
\IfFileExists{xurl.sty}{\usepackage{xurl}}{} % add URL line breaks if available
\IfFileExists{bookmark.sty}{\usepackage{bookmark}}{\usepackage{hyperref}}
\hypersetup{
  pdftitle={Maestría en Economía: Econometría},
  colorlinks=true,
  linkcolor=Maroon,
  filecolor=Maroon,
  citecolor=Blue,
  urlcolor=blue,
  pdfcreator={LaTeX via pandoc}}
\urlstyle{same} % disable monospaced font for URLs
\usepackage[margin=1in]{geometry}
\usepackage{longtable,booktabs}
% Correct order of tables after \paragraph or \subparagraph
\usepackage{etoolbox}
\makeatletter
\patchcmd\longtable{\par}{\if@noskipsec\mbox{}\fi\par}{}{}
\makeatother
% Allow footnotes in longtable head/foot
\IfFileExists{footnotehyper.sty}{\usepackage{footnotehyper}}{\usepackage{footnote}}
\makesavenoteenv{longtable}
\usepackage{graphicx}
\makeatletter
\def\maxwidth{\ifdim\Gin@nat@width>\linewidth\linewidth\else\Gin@nat@width\fi}
\def\maxheight{\ifdim\Gin@nat@height>\textheight\textheight\else\Gin@nat@height\fi}
\makeatother
% Scale images if necessary, so that they will not overflow the page
% margins by default, and it is still possible to overwrite the defaults
% using explicit options in \includegraphics[width, height, ...]{}
\setkeys{Gin}{width=\maxwidth,height=\maxheight,keepaspectratio}
% Set default figure placement to htbp
\makeatletter
\def\fps@figure{htbp}
\makeatother
\setlength{\emergencystretch}{3em} % prevent overfull lines
\providecommand{\tightlist}{%
  \setlength{\itemsep}{0pt}\setlength{\parskip}{0pt}}
\setcounter{secnumdepth}{-\maxdimen} % remove section numbering
%% See: https://bookdown.org/yihui/rmarkdown-cookbook/multi-column.html
%% I've made some additional adjustments based on my own preferences (e.g. cols
%% should be top-aligned in case of uneven vertical length)
\renewcommand{\contentsname}{Contenido}
\newenvironment{columns}[1][]{}{}

\newenvironment{column}[1]{\begin{minipage}[t]{#1}\ignorespaces}{%
\end{minipage}
\ifhmode\unskip\fi
\aftergroup\useignorespacesandallpars
}

\def\useignorespacesandallpars#1\ignorespaces\fi{%
#1\fi\ignorespacesandallpars}

\makeatletter
\def\ignorespacesandallpars{%
  \@ifnextchar\par
    {\expandafter\ignorespacesandallpars\@gobble}%
    {}%
}
\makeatother
\usepackage{booktabs}
\usepackage{threeparttable}
\usepackage{float}
\usepackage{longtable}

\title{Maestría en Economía: Econometría}
\usepackage{etoolbox}
\makeatletter
\providecommand{\subtitle}[1]{% add subtitle to \maketitle
  \apptocmd{\@title}{\par {\large #1 \par}}{}{}
}
\makeatother
\subtitle{Modelo RTC}
\usepackage{authblk}
                                        \author[]{Carlos A. Yanes
Guerra}
                                                            \affil{Universidad
del Norte \textbar{} Departamento de Economía}
                                            \date{}

\begin{document}
\maketitle

{
\hypersetup{linkcolor=}
\setcounter{tocdepth}{3}
\tableofcontents
}
\subsection{Descripción}\label{descripciuxf3n}

La idea es trabajar con la base de datos del programa Progresa del año
98 al 99 implementado en Mexico y que abordó varias variables de
ingreso, nivel de pobreza y educación. Puede descargarla desde el enlace
que se proporciona a continuación. Imagine que el interés del
tratamiento son los beneficiarios del \textbf{programa} PROGRESA en 1998
y el resultado (esperado) es una medida del cambio en los ingresos. Para
este caso, se debe considerar el año 1997 como línea de base. En este
ejercicio se asignaría a los hogares el estado de tratamiento y control
con diferentes técnicas de
aleatorización\footnote{Esta parte puede variar entre grupos de trabajo.}.

\subsection{Preguntas}\label{preguntas}

\begin{enumerate}
\def\labelenumi{\arabic{enumi}.}
\item
  Realice una breve descripción del programa Progresa. Objetivo, a quien
  iba dirigido, etc. Luego piense que se va a tratar solo el 40 \% de
  las observaciones para el año de 1998 bajo un diseño clásico
  aleatorio. Establezca un sorteo y plantee la distribución del
  tratamiento por individuo. ¿Difiere mucho de los individuos que en
  realidad fueron seleccionados? (\emph{Muestre el encabezado del sorteo
  como resultado y explique concretamente en que se basó su sorteo}.)
\item
  Elabore un análisis estadístico de características entre tratados y no
  tratados ¿Considera que existe un balance de estas características?.
  (\emph{Construya una tabla estadística por grupo, compare las
  variables Ingreso, tamaño de la familia, edad y el status de
  pobreza}).
\item
  Establezca un impacto usando primero la variable de tratados (D) y
  luego con los que fueron tratados en realidad \((D_{HH})\) sobre la
  variable ingresos (\textbf{IncomeLab}). (\emph{Realice una prueba
  estadística de T-Student entre los distintos individuos}).
\item
  Ahora haga uso del M.C.O y estime el modelo de regresión objetivo.
  Luego añada 2 o 3 controles y realice un segundo modelo.
  (\emph{Realice las regresiones correspondientes y compare en una tabla
  de salida de modelo los resultados encontrados})
\item
  Realice una prueba de identificación de efectos spillovers para esta
  parte. ¿Qué encuentra, qué puede deducir?. (\emph{Realice primero un T
  de student con los no tratados con una sola variable}). Luego
  establezca una regresión con un solo control y luego con sus controles
  respectivos
  \footnote{En esta parte cada grupo puede seleccionar los controles adecuados y justificar su uso.}
  y analice su respuesta.)
\end{enumerate}

\newpage

\subsection{Anexo}\label{anexo}

La descripción de las variables son las siguientes:

\subsubsection{Información General}\label{informaciuxf3n-general}

\begin{longtable}[]{@{}
  >{\raggedright\arraybackslash}p{(\columnwidth - 2\tabcolsep) * \real{0.6111}}
  >{\raggedright\arraybackslash}p{(\columnwidth - 2\tabcolsep) * \real{0.3889}}@{}}
\toprule\noalign{}
\begin{minipage}[b]{\linewidth}\raggedright
\textbf{Nombre variable}
\end{minipage} & \begin{minipage}[b]{\linewidth}\raggedright
\textbf{Etiqueta}
\end{minipage} \\
\midrule\noalign{}
\endhead
\bottomrule\noalign{}
\endlastfoot
\texttt{year} & time \\
\texttt{villid} & Village ID \\
\texttt{geopolid} & Federal entity \\
\texttt{hogid} & Household ID \\
\texttt{pov\_HH} & HH Poverty Status: 1= poor, 0= Non poor \\
\texttt{indexpov\_HH} & HH Poverty score/Marginalization Index
(CONAPO) \\
\texttt{D} & Village-Level Treatment status \\
\texttt{D\_HH} & Household-Level Treatment status \\
\end{longtable}

\subsubsection{Datos Individuales}\label{datos-individuales}

\begin{longtable}[]{@{}ll@{}}
\toprule\noalign{}
\textbf{Nombre variable} & \textbf{Etiqueta} \\
\midrule\noalign{}
\endhead
\bottomrule\noalign{}
\endlastfoot
\texttt{iid} & Individual ID \\
\texttt{age} & Age: Years \\
\texttt{sex} & Gender: 1= Male, 0= Female \\
\texttt{edu} & Education: Years \\
\texttt{edu\_child} & Education (6-16): Years \\
\texttt{enroll} & enroll: 1= Y, 0= N \\
\texttt{enroll\_child} & enroll child: 1= Y, 0= N \\
\texttt{labor} & labor condition: 1= occupied; 0= unoccupied \\
\texttt{salaried} & Worker: 1= Salaried; 0= unsalaried \\
\texttt{IncomeLab} & Primary Monthly Income: Pesos \\
\texttt{Income\_HH\_per} & Income per Household Member \\
\texttt{Income\_HH} & Household Income \\
\texttt{IncomeLab\_HH} & Labor Income of Household \\
\texttt{IncomeOth\_HH} & Other Household Income \\
\texttt{IncomeOth} & Other Income \\
\texttt{sick\_child} & Sick child (\textless5): 1= Y, 0= N \\
\texttt{sick} & Sick (\textgreater6): 1= Y, 0= N \\
\texttt{Measure} & 6 Months-measured child (\textless2): 1= Y, 0= N \\
\texttt{famsize} & HH size \\
\texttt{HH} & HH head \\
\end{longtable}

\subsubsection{Programas de Apoyo
Familiar}\label{programas-de-apoyo-familiar}

\begin{longtable}[]{@{}ll@{}}
\toprule\noalign{}
\textbf{Nombre variable} & \textbf{Etiqueta} \\
\midrule\noalign{}
\endhead
\bottomrule\noalign{}
\endlastfoot
\texttt{HHProg1} & HH: Despensa DIF \\
\texttt{HHProg2} & HH: Nino Solidaridad \\
\texttt{HHProg3} & HH: INI \\
\texttt{HHProg4} & HH: Beca PROBECAT o CIMO \\
\texttt{HHProg5} & HH: Programa de Empleo Temporal \\
\texttt{HHProg6} & HH: Desayuno Escolar \\
\texttt{HHProg7} & HH: Tortilla Solidaridad \\
\texttt{HHProg8} & HH: Leche de Liconsa o Conasupo \\
\texttt{HHProg9} & HH: Procampo \\
\texttt{HHProg10} & HH: Beca PROGRESA \\
\texttt{HHProg11} & HH: Papilla PROGRESA \\
\texttt{HHProg12} & HH: Apoyo Monetario PROGRESA \\
\end{longtable}

\subsubsection{Programas de Apoyo
Individual}\label{programas-de-apoyo-individual}

\begin{longtable}[]{@{}ll@{}}
\toprule\noalign{}
\textbf{Nombre variable} & \textbf{Etiqueta} \\
\midrule\noalign{}
\endhead
\bottomrule\noalign{}
\endlastfoot
\texttt{iiProg1} & ii: Despensa DIF \\
\texttt{iiProg2} & ii: Nino Solidaridad \\
\texttt{iiProg3} & ii: INI \\
\texttt{iiProg4} & ii: Beca PROBECAT o CIMO \\
\texttt{iiProg5} & ii: Programa de Empleo Temporal \\
\texttt{iiProg6} & ii: Desayuno Escolar \\
\texttt{iiProg7} & ii: Tortilla Solidaridad \\
\texttt{iiProg8} & ii: Leche de Liconsa o Conasupo \\
\texttt{iiProg9} & ii: Procampo \\
\texttt{iiProg10} & ii: Beca PROGRESA \\
\texttt{iiProg11} & ii: Papilla PROGRESA \\
\texttt{iiProg12} & ii: Apoyo Monetario PROGRESA \\
\end{longtable}

\subsubsection{Montos Mensuales}\label{montos-mensuales}

\begin{longtable}[]{@{}ll@{}}
\toprule\noalign{}
\textbf{Nombre variable} & \textbf{Etiqueta} \\
\midrule\noalign{}
\endhead
\bottomrule\noalign{}
\endlastfoot
\texttt{AmountProg2} & Montly Amount: Nino Solidaridad \\
\texttt{AmountProg3} & Montly Amount: INI \\
\texttt{AmountProg4} & Montly Amount: Beca PROBECAT o CIMO \\
\texttt{AmountProg5} & Montly Amount: Programa de Empleo Temporal \\
\texttt{AmountProg9} & Montly Amount: Procampo \\
\end{longtable}

\subsubsection{Características del Jefe de
Hogar}\label{caracteruxedsticas-del-jefe-de-hogar}

\begin{longtable}[]{@{}
  >{\raggedright\arraybackslash}p{(\columnwidth - 2\tabcolsep) * \real{0.6111}}
  >{\raggedright\arraybackslash}p{(\columnwidth - 2\tabcolsep) * \real{0.3889}}@{}}
\toprule\noalign{}
\begin{minipage}[b]{\linewidth}\raggedright
\textbf{Nombre variable}
\end{minipage} & \begin{minipage}[b]{\linewidth}\raggedright
\textbf{Etiqueta}
\end{minipage} \\
\midrule\noalign{}
\endhead
\bottomrule\noalign{}
\endlastfoot
\texttt{langhead} & Language of HH head: 1= Indigenous, 0= Non
Indigenous \\
\texttt{eduhead} & Education of HH: Years \\
\texttt{agehead} & Age of HH head: years \\
\texttt{sexhead} & Gender of HH head: 1= M, 0= F \\
\end{longtable}

\subsubsection{Otras Variables}\label{otras-variables}

\begin{longtable}[]{@{}ll@{}}
\toprule\noalign{}
\textbf{Nombre variable} & \textbf{Etiqueta} \\
\midrule\noalign{}
\endhead
\bottomrule\noalign{}
\endlastfoot
\texttt{halfyear} & Medio año \\
\texttt{p20} & nivel de escolaridad \\
\texttt{lang} & Language: 1= Indigenous, 0= Non Indigenous \\
\texttt{child} & child (6-16): 1= Y, 0= N \\
\texttt{indexpov97\_HH} & HH Poverty score 1997 \\
\texttt{indexpov03\_HH} & HH Poverty score 2003 \\
\texttt{pov03\_HH} & HH Poverty Status 2003: 1= poor, 0= Non poor \\
\texttt{condition\_ii} & Condition of residence \\
\texttt{iid2} & Individual ID/2000-2007 \\
\end{longtable}

\end{document}